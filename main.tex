\documentclass{article}
\usepackage{graphicx} % Required for inserting images
\usepackage{amsmath}
\usepackage{pifont}
\usepackage{hyperref}
\usepackage{geometry}

\geometry{a4paper,scale=0.81}

\title{Proposal of CSIT6000R NLP Group Project}
\author{
    Group No.7
}
\date{}

\begin{document}

\maketitle

\section{Paper}

\noindent

Our group project aims to firstly reproduce the framework proposed in the paper: \href{https://aclanthology.org/2023.findings-acl.59/}{\textbf{AQE: Argument Quadruplet Extraction via a Quad-Tagging Augmented Generative Approach} [Guo et al., ACL’23]}. Besides, we are looking forward to leveraging this framework in other domains.

\section{Introduction}

\noindent

Argument Mining (AM) has emerged as a crucial research area in natural language processing (NLP) and computational linguistics. It aims to identify, extract, and analyze arguments from textual data to gain deeper insights into various domains such as social media discussions, political debates, and scientific literature. While significant progress has been made in AM, the existing approaches primarily focus on addressing individual subtasks or subsets of subtasks, resulting in limitations in capturing the full complexity and richness of arguments. Therefore, this paper proposes a comprehensive task called Argument Quadruplet Extraction (AQE) and an augmentation approach called Quad-Tagging Augmented Generative approach (QuadTAG) to get more accurate results of AM.

\section{Insight \& Importance}

\noindent

The insight driving this proposal is the introduction of the Argument Quadruplet Extraction (AQE) task, which aims to extract and analyze argument quadruplets consisting of claim, evidence, stance, and evidence type from textual data. This innovative task goes beyond traditional argument mining approaches by incorporating evidence type classification. By focusing on the complete argument structure and incorporating evidence type, AQE provides a more comprehensive understanding of arguments and enables more accurate analysis.

Besides, the paper proposes Quad-Tagging Augmented Generative approach (QuadTAG) that augments the training process for AQE. QuadTAG focuses on learning the dependencies among sentences within an argument quadruplet to enhance the model.

\section{Workload}

\noindent

To successfully accomplish our project goals, we outline the following workload distribution: 

(1) \textbf{AM Prerequisites}: Each team member will individually study and gain a thorough understanding of the foundational concepts and techniques in argument mining.

(2) \textbf{IAM and QAM Dataset}: We will learn about the IAM dataset and how the authors extend it to create the QAM dataset.

(3) \textbf{Formulate AQE and QuadTAG}: Working collectively, we will reproduce the specifics of the AQE task and QuadTAG based on the proposed framework and source code.

(4) \textbf{Model Training}: Collaboratively, we will finetune the pre-trained T5-base model using the QAM dataset and the QuadTAG module based on the AQE task.

(5) \textbf{Performance analysis}: As a group, we will evaluate the performance of our trained model by conducting experiments and analyzing the results.

(6) \textbf{Possible Future Directions}: In the final phase of the project, we look forward to applying the proposed framework in other domains, like sentiment analysis.

By distributing the workload in this manner, we aim to ensure equal participation, collaborative learning, and efficient progress towards achieving our project objectives. Regular team meetings will be held to discuss progress, address challenges, and facilitate knowledge sharing among team members.


\end{document}
